\documentclass{beamer}
\usepackage{xcolor}

\begin{document}
\begin{frame}
 * Explicación de mi proyecto


En primer lugar utilicé la función Query, esta función recibe una cadena de búsqueda (query) y devuelve un objeto de tipo SearchResult. Este objeto contiene una lista de resultados de búsqueda (SearchItem) y una cadena de texto vacía.  Primero, verifica si la variable booleana estática (cargado) es falsa. Si es así, significa que los archivos de texto aún no se han cargado. Luego dentro del condicional, se obtiene un array de rutas de archivos en el directorio (content) utilizando el método (Directory.EnumerateFiles(content).ToArray()), después se inicializa un array llamado (archivos) de objetos (TFIDF) con la misma longitud que el array de rutas de archivos.  A continuación, se recorre el array de rutas de archivos y se crea un objeto (TFIDF) correspondiente para cada ruta, que se almacena en el array (archivos).  Después de cargar los archivos, la función divide la cadena de búsqueda en palabras individuales, utilizando el espacio como 
separador, y almacena las palabras en un array llamado (palabrasQuery), luego se inicializa un array de objetos 
(SearchItem) llamado (items) con la misma longitud que el array (archivos).  A continuación, se itera sobre el array 
\end{frame}
\begin{frame}
(archivos) y se crea un objeto (SearchItem) para cada archivo. El objeto (SearchItem) contiene el nombre del archivo, un fragmento de texto (Snippet) y un puntaje (Score) calculado utilizando la función (Score). Después de crear todos los objetos (SearchItem), se llama a la función (Ordenar) para ordenar los elementos en el array (items) de mayor a menor según el puntaje. Luego, se crea una nueva lista de objetos (SearchItem) llamada (arr). Luego  se itera sobre el array (items) y se agregan los objetos (SearchItem) que tienen un puntaje diferente de cero a la lista (arr).Después se copian los elementos de la lista (arr) al array (answer). Y finalmente, se crea un nuevo objeto (SearchResult) con el array (answer) y una cadena de texto vacía como argumentos y se devuelve como resultado de la función (Query). En resumen, este código realiza una búsqueda utilizando una implementación del algoritmo de Ranking TF-IDF. Carga los archivos de texto una vez (si no se han cargado) y calcula el puntaje de cada archivo según la similitud con la consulta de búsqueda. Luego, ordena los resultados por puntaje y devuelve una lista de objetos `SearchItem` que contienen información relevante del archivo correspondiente a la búsqueda.

\end{frame}
\end{document}